\section{Problem Statement}
Composition of effective and accurate time series forecasting and prediction strategies is difficult, partially due to the large number of approaches and tools available, including classical statistical methods, indicator-based quantitative models, and the more modern machine learning approach. Each tool provides certain advantages, but also comes with their own disadvantages. To maximize the gain of the advantages of each of the selected methodologies, while minimizing or alleviating the disadvantages, a way to thoughtfully and fruitfully structure hybrid strategies would possibly provide this quality.
 
Hybrid strategies have emerged as a promising methodology for time series prediction and forecasting, but as of yet remain largely unutilized \todo{source?}. Without structure, however, strategies become difficult to compare, gauge, and generalize. In order to compose and uniformly structure these strategies, it is advantageous to consider them DAGs, where nodes correspond to transformations or decisions based on the information flow of the edges between nodes of the graph. Furthermore, constructing strategies as DAGs also allows for sequential execution of strategies, enabling frequent and precise evaluation of strategy accuracy on certain sets of data.

Concisely, the problem statement for this project should reflect the ability to structure strategies in a modular manner, in which the composed strategies may utilize hybrid methodologies, and evaluate their performance during sequential execution over a set of data. Thus, \textit{the project aims to design a modular framework for composing structured hybrid time series prediction strategies, consisting of deterministic and machine learning components, and evaluating their performance through sequential execution on historical data.}