\section{Related Work}\label{sec:cunt}

The engineering of automated trading systems has historically presented a high barrier to entry, requiring a synthesis of advanced software architecture, low-latency data processing, and complex domain logic. Traditionally, translating a trading hypothesis into a functional system involved writing extensive boilerplate code to manage data feeds, synchronization, and order execution \cite{huang2019automated}. This technical debt often creates a disconnect between the design of a strategy and its implementation. Consequently, the field has seen a shift toward visual programming and low-code solutions, aiming to abstract these engineering complexities into modular, reusable components \cite{skotnica2019das}.

In terms of strategy logic, the field is generally divided into two forms. The first relies on technical analysis, where traders use mathematical formulas like Moving Averages or RSI to find trends. These methods are transparent and easy to understand but can be rigid \cite{lo2000foundations}. The second form utilizes machine learning, particularly deep learning models like LSTMs, which are excellent at finding hidden patterns in data but often act as "black boxes" that are hard to interpret \cite{sezer2020financial}.

Recent research suggests that the most effective approach is not to choose one over the other, but to combine them. By using simple logic rules to filter or confirm the complex predictions made by AI models, traders can create "hybrid" strategies that are both powerful and safer to deploy \cite{vanbekkum2021modular}. However, most existing software platforms force users to pick a side: they are either simple rule-based builders or complex code-first ML frameworks. There is a lack of tools that seamlessly integrate these two paradigms in a visual, user-friendly manner.