\section{Related Work}

To understand the motivation for a modular hybrid framework, it is necessary to review the existing barriers in both software engineering and strategy design. This section outlines the shift from code-heavy implementations to visual abstractions, and compares the strengths and weaknesses of statistical versus machine learning models, laying the groundwork for the hybrid approach.

\subsection{Visual Programming and Low-Code Solutions}
To tackle these problems, the field has seen a shift toward visual programming and low-code solutions. These tools aim to hide the complex code behind modular, reusable components.

For instance, visual languages have been successfully used to model complex logic like blockchain smart contracts~\cite{skotnica2019das}. Instead of writing code, a user can arrange visual blocks to define the contract, making it easier to see how the logic flows. Similarly, flow-based programming has proven effective for orchestrating complex data pipelines~\cite{paleyes2022empirical}. By treating a system as a directed graph—where data flows between nodes—users can intuitively compose logic without managing low-level code. In a trading context, this allows for the visual assembly of a strategy where for example pre-trained machine learning signals are treated as simple nodes.

\subsection{Quantitative vs. Machine Learning Approaches}
Regarding the strategy logic itself, the field is generally split into two forms. 

\textbf{Technical Analysis:} The first relies on technical analysis, using mathematical formulas and statistical indicators to identify historical patterns~\cite{lo2000foundations}. These methods are transparent and easy to interpret—for example, a simple rule like `buy when the moving average crosses this line`. However, these linear methods can be rigid and often fail to capture the complex dynamics of modern markets~\cite{huang2019automated}.

\textbf{Deep Learning:} The second form utilizes machine learning. Deep learning models, particularly LSTM networks, excel at finding hidden, non-linear patterns in data that standard formulas miss~\cite{sezer2020financial}. The downside is that these models often act as ``black boxes``. They might give an accurate prediction, but they offer little insight into \textit{why} that decision was made.

\subsection{Hybrid Strategies}
Recent research suggests that the most effective approach is not to choose one over the other, but to combine them. Theoretically, hybrid models are superior because they allow for the modeling of both linear (statistical) and non-linear (neural) patterns simultaneously, a capability that single models lack~\cite{zhang2003time}.

These `hybrid systems` integrate data-driven components (like AI) with knowledge-driven components (like logical rules)~\cite{vanbekkum2021modular}. In a practical trading context, this combination has been shown to outperform standalone strategies; optimizing technical analysis strategies with machine learning not only improves the profitability of trading signals but significantly reduces the risk associated with the strategy~\cite{ayala2021technical}.

By using simple logic rules to filter or confirm the complex predictions made by AI models, traders can create strategies that are safer to deploy. For instance, a user might use an LSTM to predict a trend but use a standard logic rule to block the trade if market volatility is too high. However, a gap remains in the tools available to build these systems. Platforms like MetaTrader or TradingView offer accessible, visual tools for rule-based strategies but lack deep integration for modern AI. Conversely, frameworks like TensorFlow or PyTorch offer immense predictive power but require significant "glue code" and engineering effort to deploy~\cite{sculley2015hidden}. Consequently, there is a distinct lack of tools that seamlessly integrate these two worlds in a visual, user-friendly manner.