\section{Conclusion}\label{sec:conclusion}

While many factors contribute to the results of the comparative tests of the strategies, an overall positive result was observed. Furthermore, as discussed in section \ref{discussion}, a contribution of the project is the ability to introduce machine-learning components as forecasting or indicating nodes into time series prediction and forecasting strategies, allowing them to function in coherence with traditional statistical models. Although, there still exists many external contributing factors, which may be hard to capture the temporal indication of using strategies like these. These external factors are very prominent in financial markets, where geopolitics and macroeconomics play major roles in ordinary financial analysis. This is especially apparent within the test case of BTC, as discussed in section \ref{datasets}, where the market is very sentiment-driven, creating a highly volatile domain.

Utilizing a node-based strategy composition framework along with a modular API communication strategy allowed for efficient development and deployment pipelines with minimal conflict during continuous deployment. This approach ensured continued functionality of the various individual components as they were each developed and adjusted throughout the project. Furthermore, being able to build and test strategies within a rule-based framework, where minor adjustments could quickly be made to strategies, allowed for a an environment where models and strategies could frequently be evaluated and improved. Not only did this aid the development process, but would also provide a significant advantage to potential users of such a platform. The rapid composition, evaluation and recomposition of strategies support the conclusion, that a modular and experimental tool such as a node-based strategy building platform would suit time series forecasting tasks, not only during strategy development, but also strategy adjustment and maintenance.

The comparative analysis of the Bitcoin and JPM cases makes it clear that the effectiveness and accuracy of ML-driven time series prediction and forecasting within a modular strategy context is fundamentally affected by the target asset of the strategy. While the ML-only baseline experiments showed a favorable performance on Bitcoin, it was apparent that it is largely market-specific. In contrast, the JPM experiments showed that ML-models perform well with appropriate statistical indicators in market-agnostic scenarios given a less volatile environment. The superior win rates and lower drawdowns of the hybrid model performances indicate an ability to create more stable and generalizable strategies, which can perform well in forecasting short term, low volatility assets.

Looking ahead, this conclusion also opens possibilities for other directions of further study and development. While only financial data in the form of stocks and crypto-currencies were evaluated during the project, many other markets exist, financial and non-financial, as mentioned in \ref{introduction}. Exploring alternative machine-learning architectures of varying complexities could also prove beneficial to the project.

Overall, the results proved fruitful in demonstrating the value in utilizing smaller machine-learning models as forecasting components in a broader statistical and quantitative analysis system. The flexibility provided by the node-based strategy composition, which can utilize both statistical and machine-learning models, along with the interpretability of a backtesting unit for real-time evaluation of composed strategies, comes together to form a system, which has been shown to be functional on financial data, and lays the foundation for further study and development.