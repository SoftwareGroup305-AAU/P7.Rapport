\section{Conclusion}\label{sec:conclusion}

While many factors contribute to the results of the comparative tests of the strategies, an overall positive result was observed. Furthermore, as discussed in section \ref{discussion}, a contribution of the project is the ability to introduce machine-learning components as forecasting or indicating nodes into time series prediction and forecasting strategies, allowing them to function in coherence with traditional statistical models. 

Utilizing a node-based strategy composition framework along with a modular API communication strategy allowed for efficient development and deployment pipelines with minimal conflict during continuous deployment. This approach ensured continued functionality of the various individual components as they were each developed and adjusted throughout the project. Furthermore, being able to build and test strategies within a rule-based framework, where minor adjustments could quickly be made to strategies, allowed for a an environment where models and strategies could frequently be evaluated and improved. Not only did this aid the development process, but would also provide a significant advantage to potential users of such a platform.

